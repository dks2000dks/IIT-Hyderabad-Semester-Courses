\begin{enumerate}[label=\thesubsection.\arabic*.,ref=\thesubsection.\theenumi]
\numberwithin{equation}{enumi}

\item For the feedback current amplifier shown in \ref{fig:Input}, Draw the Small-Signal Model.eglect the Early effect in $Q_{1}$ and $Q_{2}$.\\
\begin{figure}[h!]
	\begin{center}
		\resizebox{\columnwidth/1}{!}{\begin{circuitikz}[american]
\ctikzset{tripoles/mos style/arrows}
\draw  (0,0) node[ground](GND){} -- (0,1) to[isource, l= $I_{s}$] (0,3) -- (2,3) to[R=$R_{F}$, i=$I_{f}$] (4,3) -- (6,3)node[label={right:C}]{} to[R=$R_{m}$] (6,0) node[ground](GND){} (6,0);

\draw (1,5) node[nmos,](Q1){};
\draw (1,3) node[label={below:A}]{} to[short,i=$I_{i}$] (Q1.S);
\draw (Q1.center) node[right]{{$Q_{1}$}};
\draw (Q1.G) -- (0,5) node[ground](GND){};
\draw (1,9) node[vcc](VCC){$V_{CC}$} (1,9);
\draw  (1,6)node[label={left:B}]{} -- (1,6.5) to[R=$R_{D}$, i=$I_{i}$] (1,9);
\draw (Q1.D) -- (1,6);

\draw (6,6)node[pmos,](Q2){};
\draw (Q2.D) to[R=$R_{L}$,i=$I_{o}$] (6,3);
\draw (Q2.center) node[right]{{$Q_{2}$}};
\draw (Q2.G) -- (1,6);
\draw (Q2.S) -- (6,9) node[vcc](VCC){$V_{CC}$} (6,10);
\end{circuitikz}
}
	\end{center}
	\caption{}
	\label{fig:Input}
\end{figure}

\solution
While drawing a Small-Signal Model, we ground all constant voltage sources and open all constant current sources. All Small-Signal paramters are obtained from DC-Analysis of the circuit. Neglecting Early effect, in Small-Signal Analysis a N-MOSFET is modelled as a Current Source with value of current equal to $g_{m}v_{gs}$ flowing from Drain to Source. Whereas a P-MOSFET is modelled as a Current Source with value of current equal to $g_{m}v_{sg}$ flowing from Source to Drain.
\begin{figure}[h!]
	\begin{center}
		\resizebox{\columnwidth/1}{!}{\begin{circuitikz}[american]
\ctikzset{tripoles/mos style/arrows}
\draw  (0,0) node[ground](GND){} -- (0,1) to[isource, l= $I_{s}$] (0,3) -- (2,3) to[R=$R_{F}$, i=$I_{f}$] (4,3) -- (6,3)node[label={right:C}]{} to[R=$R_{m}$] (6,0) node[ground](GND){} (6,0);

\draw (1,3) node[label={below:A}]{} to[short,i=$I_{i}$] (1,4);
\draw (1,6) node[label={right:B}]{} to[cisource, l= $-g_{m_{1}}v_{A}$] (1,4);
\draw (1,4) to[short, -o] (-1,4) node[label={above:$-$}]{} node[label={below:$S_{1}$}]{};
\draw (-1,5.5) node[label={below:$-v_{A}$}]{};
\draw (-2,6) node[label={below:$G_{1}$}]{} to[short, -o] (-1,6) node[label={below:$+$}]{};
\draw (1,6) to[R=$R_{D}$, i=$I_{i}$] (1,9);
\draw (1,9) node[ground,rotate=180](GND){} (1,9);

\draw (6,7) to[R=$R_{L}$,i=$I_{o}$] (6,3);
\draw (6,8) to[cisource, l= $-g_{m_{2}}v_{B}$] (6,7);
\draw (6,6.5) to[short, -o] (4,6.5) node[label={above:$-$}]{} node[label={below:$G_{2}$}]{};
\draw (4,8) node[label={below:$-v_{B}$}]{};
\draw (3,8.5) node[label={below:$S_{2}$}]{} to[short, -o] (4,8.5) node[label={below:$+$}]{};
\draw (6,7) -- (6,9);
\draw (6,9) node[ground,rotate=180](GND){} (6,9);

\end{circuitikz}
}
	\end{center}
	\caption{Small Signal Model}
	\label{fig:Small_Signal}
\end{figure}

%------------------------------------------------------------------------%

\item Describe how the given circuit is a Negetive Feedback Current Amplifier.\\
\solution 
For the feedback to be negative, $I_{f}$ must have the same polarity as $I_{s}$. To ascertain that this is the case, we assume an increase in $I_{s}$ and follow the change around the loop: An increase in $I_{s}$ causes $I_{i}$ to increase and the drain voltage of $Q_{1}$ will increase. Since this voltage is applied to the gate of the p-channel device $Q_{2}$ , its increase will cause $I_{o}$ , the drain current of $Q_{2}$, to decrease. Thus, the voltage across $R_{M}$ will decrease, which will cause $I_{f}$ to increase. This is the same polarity assumed for the initial change in $I_{s}$, verifying that the feedback is indeed negative.\\
%------------------------------------------------------------------------%

\item Write all KVL and KCL Equations\\
\solution
\begin{align}
I_{i} = g_{m_{1}}v_{A}\\
v_{B} = I_{i}R_{D}\\
I_{o} = -g_{m_{2}}v_{B}\\
v_{A} = I_{F}R_{F} - (I_{F} + I_{o})R_{M}
\end{align}
%------------------------------------------------------------------------%

\item Find the Expression for the Open-Loop Gain $G=\frac{I_{o}}{I_{i}}$, from the Small-Signal Model. For simplicity, neglect the Early effect in $Q_{1}$ and $Q_{2}$.\\
\solution
In Small-Signal Model,
\begin{align}
v_{B} = I_{i}R_{D}\\
v_{gs_{2}} = v_{B} = I_{i}R_{D}
\end{align}

In Small-Signal Analysis, P-MOSFET is modelled as a current source where current flows from Source to Drain. So, the value of current flowing from Source to Drain in P-MOSFET is,
\begin{align}
I_{o} =  -g_{m_{2}}v_{gs_{2}} = -g_{m_{2}}I_{i}R_{D}
\end{align}
So, the Open-Circuit Gain is
\begin{align}
G = \frac{I_{o}}{I_{i}} =  -g_{m_{2}}R_{D}
\end{align}
%------------------------------------------------------------------------%

\item Find the Expression of the Feedback Factor $H = \frac{I_{f}}{I_{o}}$, from Small-Signal Model. For simplicity, neglect the Early effect in $Q_{1}$ and $Q_{2}$.\\
\solution\\
$I_{o}$ is fed to a current divider formed by $R_{M}$ and $R_{F}$. Assuming system has Good Gain, Most part of $I_{s}$ flows as $I_{F}$ leaving behind small $I_{i}$. As $I_{i}$ is small, the voltage at point 'A' is very small and is considered, $v_{A} \simeq 0$. So $R_{F}$ and $R_{M}$ are parallel and Voltage Drop across them is same.

From (2.1.3.4),
\begin{align}
(I_{o} + I_{f})R_{M} \simeq -I_{f}R_{F}\\
\frac{I_{f}}{I_{o}} \simeq -\frac{R_{M}}{R_{F}+R_{M}}
\end{align}
So, the Feedback Factor,
\begin{align}
H \equiv \frac{I_{f}}{I_{o}} \simeq-\frac{R_{M}}{R_{F}+R_{M}}
\end{align}
%------------------------------------------------------------------------%
\item Find the Expression for the Closed-Loop Gain $T=\frac{I_{o}}{I_{s}}$. For simplicity, neglect the Early effect in $Q_{1}$ and $Q_{2}$.\\
\solution \\
From Open-Loop Gain and Feedback Factor,
\begin{align}
I_{s} = I_{i} + I_{f}\\
I_{s} = \frac{I_{o}}{G} + H I_{o}\\
GI_{s} = I_{o}(1+GH)\\
\frac{I_{o}}{I_{s}} = \frac{G}{1+GH}\\
\frac{I_{o}}{I_{s}}=-\frac{g_{m_{2}} R_{D}}{1+g_{m_{2}} R_{D} /\left(1+\frac{R_{F}}{R_{M}}\right)}
\end{align}

So the Block Diagram of Feedback Current Amplifier is
\begin{figure}[ht!]
	\begin{center}
		\resizebox{\columnwidth}{!}{\begin{circuitikz}[american]

\draw (2,2)  node[op amp] (OA) {};
\draw (OA.up) -- ++(0, 0.3) node[vcc]{$+10V$};
\draw (OA.down) -- ++(0,-0.3) node[vee]{$-10V$};
\draw (OA.+) -- (0,1.5) to[vsourcesin, l= $v_{s}$] (0,0) node[ground](GND){};
\draw (OA.-) -- (0,2.5) node[ground, rotate=270](GND){};
\draw (OA.out) -- (3,2) node[label={above:$v_{a}$}]{};
\draw (3,2) to[R=$R_{1}$] (5.5,2) node[label={above:$v_{b}$}]{} to[C,l_=$C_{1}$] (5.5,0) node[ground](GND){};

\draw (7.5,2.5) node[op amp] (OB) {};
\draw (OB.up) -- ++(0, 0.3) node[vcc]{$+10V$};
\draw (OB.down) -- ++(0,-0.3) node[vee]{$-10V$};
\draw (OB.+) -- (5.5,2);
\draw (OB.-) -- ++(-0.5,0) node[ground,rotate=270](GND){};;
\draw (OB.out) to[R=$R_{2}$] (10.5,2.5) node[label={above:$v_{c}$}]{} to[C,l_=$C_{2}$] (10.5,0) node[ground](GND){};

\draw (12.5,3) node[op amp] (OC) {};
\draw (OC.up) -- ++(0, 0.3) node[vcc]{$+10V$};
\draw (OC.down) -- ++(0,-0.3) node[vee]{$-10V$};
\draw (OC.+) -- (10.5,2.5);
\draw (OC.-) -- ++(-0.5,0) node[ground,rotate=270](GND){};;
\draw (OC.out) to[R=$R_{3}$] (15.5,3) node[label={right:$v_{o}$}]{} to[C,l_=$C_{3}$] (15.5,0) node[ground](GND){};

\end{circuitikz}
}
	\end{center}
	\caption{}
	\label{fig:Control_System}
\end{figure}

where $G = -g_{m_{2}} R_{D}$ and $H = -\frac{R_{M}}{R_{F}+R_{M}}$\\

So, the value of Closed-Loop Gain is
\begin{align}
T = \frac{I_{o}}{I_{s}}=-\frac{g_{m_{2}} R_{D}}{1+g_{m_{2}} R_{D} /\left(1+\frac{R_{F}}{R_{M}}\right)}
\end{align}
%------------------------------------------------------------------------%

\item Draw the Circuit Diagram of Feedback Network.\\
\solution
The Circuit Diagram of Feedback Network is \ref{fig:Feedback_Network}
\begin{figure}[ht!]
	\begin{center}
		\resizebox{\columnwidth/2}{!}{\begin{circuitikz}[american]

\draw (0,0) node[ground](GND){} to[vsourcesin, l= $v_{s}$] (0,2) to[short,-o] (1,2) node[label={below:$+$}]{};
\draw (1,0) node[ground](GND){} to[short,-o] (1,0.1) node[label={above:$-$}] {};
\draw (3,2) node[label={above:$v_{a}$}]{};
\draw (1,0.625) node[label={$v_{i}$}] {};
\draw (3,0) node[ground](GND){} to[vsourcesin, l= $10^5 v_{i}$] (3,2);
\draw (3,2) to[R=$10^{2}\ohm$] (5.5,2) node[label={above:$v_{b}$}]{} to[C,l_=$\frac{10^{-9}}{2\pi}F$] (5.5,0) node[ground](GND){};
\draw (5.5,2) to[R=$10^{3}\ohm$] (8,2) node[label={above:$v_{c}$}]{} to[C,l_=$\frac{10^{-9}}{2\pi}F$] (8,0) node[ground](GND){};
\draw (8,2) to[R=$10^{4}\ohm$] (10.5,2) to[C,l_=$\frac{10^{-9}}{2\pi}F$] (10.5,0) node[ground](GND){};
\draw (10.5,2) -- (11.5,2) node[label={above:$v_{o}$}]{};

\end{circuitikz}
}
	\end{center}
	\caption{Feedback Network}
	\label{fig:Feedback_Network}
\end{figure}

By KVL and KCL,
\begin{align}
(I_{o} + I_{f})R_{M} = -I_{f}R_{F}\\
\frac{I_{f}}{I_{o}} = -\frac{R_{M}}{R_{F}+R_{M}}
\end{align}

So, Gain of Feedback Network is
\begin{align}
H = -\frac{R_{M}}{R_{F}+R_{M}}
\end{align}

The Block Diagram of Feedback Network is
\begin{figure}[ht!]
	\begin{center}
		\resizebox{\columnwidth}{!}{\begin{circuitikz}[american]
\tikzset{quad/.style={draw, minimum height=2.4cm, minimum width=4cm}}
\node[quad] (A) at (0,0) {$H$};
\draw ($(A.north west)!.175!(A.west)$) to[short,-o] ++(-2,0) -- (-5,1)
      ($(A.south west)!.175!(A.west)$) to[short,-o] ++(-2,0) -- (-5,-1)
      ($(A.north east)!.175!(A.east)$) to[short,-o] ++(1,0)
      ($(A.south east)!.175!(A.east)$) to[short,-o] ++(1,0);

\draw (-5,-1) to[short, i=$I_{f}$] (-5,1);
\draw (-5,-1) to[closing switch, o-o] (-5,-2) node[ground](GND){};

\draw (3,-1) -- (5,-1) to[isource, l= $I_{o}$] (5,1) -- (3,1);

\end{circuitikz}
}
	\end{center}
	\caption{Feedback Block Diagram}
	\label{fig:Feedback_Block}
\end{figure}
%------------------------------------------------------------------------%

\item Find $R_{11}$ and $R_{22}$  of Feedback Network where $R_{11}$ is input resistance through Port-1 and $R_{22}$ is Input Resistance through Port-2.\\
\solution\\
While calculating $R_{11}$, Port-2 should be Opened. So, Net-Resistance seen through Port-1 is 
\begin{align}
R_{11} = R_{F} + R_{M}
\end{align}
While calculating $R_{22}$, Port-1 should be Shorted. So, Net-Resistance seen through Port-2 is 
\begin{align}
R_{22} = R_{F} || R_{M}\\
R_{22} = \frac{R_{F}R_{M}}{R_{F}+R_{M}}
\end{align}

So,
\begin{align}
R_{11} = R_{F} + R_{M}\\
R_{22} = \frac{R_{F}R_{M}}{R_{F}+R_{M}}
\end{align}
%------------------------------------------------------------------------%

\item Draw the Circuit Diagram of Open-Loop Network.\\
\solution\\
As the ciruit is Shunt-Series Topology, $R_{11}$ is shunted across the Input Terminal and $R_{22}$ is added in series to Output Terminal.

Current Flowing through $R_{D}$ is approximately same as $I_{i}$, as $R_{F}$ is a Large Resistance.

The Circuit Diagram of Open-Loop Network is \ref{fig:OpenLoop_Network}
\begin{figure}[ht!]
	\begin{center}
		\resizebox{\columnwidth}{!}{\begin{circuitikz}[american]
\ctikzset{tripoles/mos style/arrows}
\draw (1,2) to[short, -o] (0,2) node[label={below:$v_{o}$}]{};
\draw (1,2) to[R=$100k\ohm$] (2,2) -- (3,2) to[R=$10\ohm$] (3,0) node[ground](GND){};
\draw (3,2) to[short, -o] (4,2) node[label={below:$v_{f}$}]{};
\end{circuitikz}
}
	\end{center}
	\caption{Open-Loop Network}
	\label{fig:OpenLoop_Network}
\end{figure}

By KVL and KCL,
\begin{align}
v_{B} = I_{i}R_{D}\\
v_{gs_{2}} = v_{B} = I_{i}R_{D}\\
I_{o} =  -g_{m_{2}}v_{gs_{2}} = -g_{m_{2}}I_{i}R_{D}\\
\frac{I_{o}}{I_{i}} = -g_{m_{2}}R_{D}
\end{align}

So, Open-Loop Gain is
\begin{align}
G = \frac{I_{o}}{I_{i}} =  -g_{m_{2}}R_{D}
\end{align}

The Block Diagram of Open-Loop Network is
\begin{figure}[ht!]
	\begin{center}
		\resizebox{\columnwidth}{!}{\begin{circuitikz}[american]
\tikzset{quad/.style={draw, minimum height=2.4cm, minimum width=4cm}}
\node[quad] (A) at (0,0) {$G$};
\draw ($(A.north west)!.175!(A.west)$) to[short] ++(-3,0)
      ($(A.south west)!.175!(A.west)$) to[short] ++(-3,0)
      ($(A.north east)!.175!(A.east)$) to[short] ++(1,0)
      ($(A.south east)!.175!(A.east)$) to[short] ++(1,0);

\draw (-5,-1) to[R,n=res1] (-5,1);
\draw (-5,-1) -- (-6,-1);
\draw (-5,1) -- (-6,1);
\draw (-6,-1) -- (-7.5,-1) to[isource, l= $I_{i}$] (-7.5,1) -- (-6,1);

\draw (3,-1) to[R,n=res2] (5,-1) -- (6,-1) to[R=$R_{L}$] (6,1) -- (3,1);
\draw (res1.s) node[right]{$R_{11} = R_{F}+R_{M}$};
\draw (res2.s) node[below]{$R_{22} = R_{F}||R_{M}$};

\end{circuitikz}
}
	\end{center}
	\caption{Open-Loop Block Diagram}
	\label{fig:OpenLoop_Block}
\end{figure}

\end{enumerate}
