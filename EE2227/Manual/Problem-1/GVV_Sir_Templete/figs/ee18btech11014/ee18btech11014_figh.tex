\tikzstyle{block} = [draw, rectangle, 
    minimum height=1.5em, minimum width=3em]
\tikzstyle{sum} = [draw, circle, node distance=1cm]
\tikzstyle{input} = [coordinate]
\tikzstyle{output} = [coordinate]
\tikzstyle{pinstyle} = [pin edge={to-,thin,black}]

% The block diagram code is probably more verbose than necessary
\begin{tikzpicture}[auto, node distance=2.5cm,>=latex']
    % We start by placing the blocks
    \node [input, name=input] {};
    \node [block, right of=input] (g) {$10^{5}$};
    \node [block, right of=g] (p1) {$\frac{1}{1+\frac{s}{2\pi \times 10^{7}}}$};
    \node [block, right of=p1] (p2) {$\frac{1}{1+\frac{s}{2\pi \times 10^{6}}}$};
    \node [block, right of=p2] (p3) {$\frac{1}{1+\frac{s}{2\pi \times 10^{5}}}$};
    \node [output, right of=p3] (output) {};

    % Once the nodes are placed, connecting them is easy. 
    \draw [draw,->] (input) -- node {$v_{s}$} (g);
    \draw [->] (g) -- node {$v_{a}$} (p1);
    \draw [->] (p1) -- node {$v_{b}$} (p2);
    \draw [->] (p2) -- node {$v_{c}$} (p3);
    \draw [->] (p3) -- node {$v_{o}$} (output);
    
\end{tikzpicture}
